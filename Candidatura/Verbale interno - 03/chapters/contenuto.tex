\section{Ordine del giorno}
\begin{itemize}
    \item Redazione domande per le aziende di interesse;
    \item Redazione domande per il docente;
    \item Condivisione delle conoscenze in merito all'utilizzo di git e LaTeX;
\end{itemize}

\section{Resoconto}
Al termine della riunione precedente e delle conversazioni avvenute nei canali di comunicazione, il gruppo ha concretizzato un elenco di domande da porre alle aziende proponenti i capitolati di interesse.
Sono state quindi inviate mail ai contatti forniti come riferimento, preannunciando alcune delle domande e chiedendo disponibilità per un colloquio.

Siccome sono sorti dubbi in merito alla metodologia di aggiudicazione degli appalti, si è scelto di contattare allo stesso modo anche il docente Tullio Vardanega.

Avendo temrinato l'analisi dei capitolati si è cominciato a suddividere il lavoro inerente la stesura dei documenti richiesti per la partecipazione alle gare d'appalto.
Per l'assegnazione dei compiti si è scelto di utilizzare gli strumenti disponibili sulla piattaforma GitHub.
Si è deciso infine di sfruttare il tempo a disposizione per avviare un processo formativo in cui sono state condivise conoscenze su argomenti quali l'utilizzo di GitHub e LaTeX, con cui alcuni membri del gruppo non avevano dimestichezza.