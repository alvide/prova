\subsection{Capitolato C4 - A ChatGPT plugin with Nuvolaris}


\subsubsection{Descrizione}
\begin{itemize}
    \item \textbf{Proponente}: \href{https://www.nuvolaris.io/}{Nuvolaris}.
    \item \textbf{Obiettivo}: creazione di un \textit{plugin} di ChatGPT che sfrutti Nuvolaris per richiedere la creazione di un'applicazione a partire da un \textit{template}, che dovrà essere subito utilizzabile.
\end{itemize}


\subsubsection{Tecnologie}
\begin{itemize}
    \item \textbf{Nuvolaris}: sistema per lo sviluppo in \textit{cloud}.
    \item \textbf{OpenAI API}: API per l'accesso ai nuovi modelli di intelligenza artificiale sviluppati da OpenAI.
\end{itemize}


\subsubsection{Considerazioni}
\begin{minipage}[t]{0.45\linewidth}
    \vspace{0pt}
    {\renewcommand{\arraystretch}{1.5}
    \begin{tabular}{p{1\linewidth}}
        \multicolumn{1}{c}{\textbf{Pro}} \\
        \midrule
        Potenziale nel poter creare applicazioni tramite ChatGPT\\
        Non è necessaria una gestione del \textit{server}\\
        \hline
    \end{tabular}
    }
\end{minipage}
\hspace{0.05\linewidth}
\begin{minipage}[t]{0.45\linewidth}
    \vspace{0pt}
    {\renewcommand{\arraystretch}{1.5}
    \begin{tabular}{p{1\linewidth}}
        \multicolumn{1}{c}{\textbf{Contro}} \\
        \midrule
        Non ha attirato l'interesse del gruppo \\
        Utilizzo di tecnologie proprietarie \\
        Analisi del progetto poco approfondita \\
        \hline
    \end{tabular}
    }
\end{minipage}
\vspace{1em}

\noindent
L'utilizzo di tecnologie moderne e diffuse come ChatGPT è stato valutato positivamente, tuttavia non essendo una richiesta esclusiva di questo capitolato e considerando che lo sviluppo avverrebbe in ambito di tecnologie proprietarie si è deciso di concentrarsi su altri progetti.