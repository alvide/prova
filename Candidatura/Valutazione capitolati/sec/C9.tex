\subsection{Capitolato C9 - ChatSQL: creare frasi SQL da linguaggio naturale}


\subsubsection{Descrizione}
\begin{itemize}
    \item \textbf{Proponente}: \href{https://www.zucchetti.it/website/cms/home.html}{Zucchetti}.
    \item \textbf{Obiettivo}: scrivere un sistema di \textit{prompt} per convertire il linguaggio naturale in un comando SQL, usando ChatGPT.
\end{itemize}


\subsubsection{Tecnologie}
\begin{itemize}
    \item \textbf{Basi di dati}: nessuna preferenza verso un particolare linguaggio SQL.
    \item \textbf{OpenAI API}: API per l'accesso ai nuovi modelli di intelligenza artificiale sviluppati da OpenAI.
\end{itemize}


\subsubsection{Considerazioni}
\begin{minipage}[t]{0.45\linewidth}
    \vspace{0pt}
    {\renewcommand{\arraystretch}{1.5}
    \begin{tabular}{p{1\linewidth}}
        \multicolumn{1}{c}{\textbf{Pro}} \\
        \midrule
        Ambiti di applicazione interessanti: semplificazione nell'utilizzo di \textit{database}\\
        Utilizzo di tecnologie moderne come ChatGPT \\
        Vari requisiti opzionali interessanti \\
        Gli obiettivi sono chiari ed in gerarchia \\
        \hline
    \end{tabular}
    }
\end{minipage}
\hspace{0.05\linewidth}
\begin{minipage}[t]{0.45\linewidth}
    \vspace{0pt}
    {\renewcommand{\arraystretch}{1.5}
    \begin{tabular}{p{1\linewidth}}
        \multicolumn{1}{c}{\textbf{Contro}} \\
        \midrule
        Il progetto ha suscitato minore interesse rispetto ad altri capitolati \\
        \hline
    \end{tabular}
    }
\end{minipage}
\vspace{1em}

\noindent
La proposta è stata valutata positivamente, ma si è comunque scelto di concentrarsi su altri progetti, vista la presenza di capitolati che hanno riscosso maggiore interesse.\\
L'utilizzo di tecnologie moderne e diffuse come ChatGPT risulta essere interessante, tuttavia non è una richiesta esclusiva di questo capitolato.