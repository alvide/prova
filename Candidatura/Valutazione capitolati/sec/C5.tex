\subsection{Capitolato C5 - Warehouse Management 3D}

\subsubsection{Descrizione}
\begin{itemize}
    \item \textbf{Proponente}: 
		\href{https://www.sanmarcoinformatica.com/}{Sanmarco Informatica};
    \item \textbf{Obiettivo}:  creazione di un'applicazione per 
		visualizzare e simulare gli spazi fisici di un magazzino, al fine di 
		monitorare le performance, migliorare lo sfruttamento degli spazi e 
		ottimizzare i processi di logistica.
\end{itemize}


\subsubsection{Tecnologie}
\begin{itemize}
    \item \textbf{Three.js}: libreria per la creazione di grafica 3D in un 
		\textit{browser web}. Tecnologie alternative:
        \begin{itemize}
            \item \textbf{Unity}: per lo sviluppo in C\#;
	        \item \textbf{Unreal Engine}: per lo sviluppo in C++.
        \end{itemize}
\end{itemize}
L'azienda ha posto maggior enfasi sull'utilizzo della libreria Three.js, con 
conseguente preferenza del linguaggio JavaScript.


\subsubsection{Considerazioni}
\begin{minipage}[t]{0.45\linewidth}
    \vspace{0pt}
    {\renewcommand{\arraystretch}{1.5}
    \begin{tabular}{p{1\linewidth}}
        \multicolumn{1}{c}{\textbf{Pro}} \\
        \midrule
		Il campo di sviluppo risulta di interesse \\
		Le tecnologie consigliate suscitano interesse \\
		Gli obiettivi sono chiari ed in gerarchia \\
		Notevole flessibilità rispetto agli obiettivi opzionali \\
		Ampia disponibilità dell'azienda a incontri con il gruppo di lavoro \\
        \hline
    \end{tabular}
    }
\end{minipage}
\hspace{0.05\linewidth}
\begin{minipage}[t]{0.45\linewidth}
    \vspace{0pt}
    {\renewcommand{\arraystretch}{1.5}
    \begin{tabular}{p{1\linewidth}}
        \multicolumn{1}{c}{\textbf{Contro}} \\
        \midrule
        \hline
    \end{tabular}
    }
\end{minipage}
\vspace{1em}

Le tecnologie proposte risultano interessanti e l'apprendimento di strumenti per 
la gestione di grafica 3D ha stimolato i membri del gruppo.
Essendo il programma pensato per essere eseguito sul web (una caratteristica che 
gli permette di essere \textit{crossplatform}). Il gruppo ha apprezzato molto la
flessibilità notevole offerta dall'azienda inerentemente agli obiettivi
opzionali, che sono scelti dal gruppo stesso e concordati con l'azienda in corso
d'opera.
