\section{Documentazione}
\subsection{Strumenti}
Gli strumenti utilizzati per la creazione dei documenti sono:
\begin{itemize}
    \item \textbf{LaTeX}: strumento per la creazione di documenti \\
        \href{https://www.latex-project.org/}{(www.latex-project.org)};
    \item \textbf{VisualStudio Code}: GUI con integrazioni per la creazione di documenti scritti in LaTeX e per la gestione delle repository git \\
        \href{https://code.visualstudio.com/}{(code.visualstudio.com)}
\end{itemize}

\subsection{Creazione e modifica di un documento}
Lo strumento utilizzato per la redazione di documenti è LaTeX, quando si crea un nuovo documento è obbligatorio l'utilizzo di uno dei template disponibili al fine di uniformare la documentazione rilasciata.
La versione di partenza deve sempre essere \texttt{0.1.0} e venire aggiornata ad ogni modifica.
I \textit{templates} disponibili sono:
\begin{itemize}
    \item \textbf{Verbali}: per i resoconti degli incontri interni al gruppo o con le aziende proponenti;
    \item \textbf{Documenti}: ogni documento che non sia un verbale, prevede obbligatoriamente la presenza di un registro delle modifiche;
    \item \textbf{Presentazioni}.
\end{itemize}
Qualora sia necessaria una diversa tipologia di documento le sue caratteristiche grafiche ed organizzative dovranno essere discusse nel gruppo.

\subsection{Ruoli di redattore, verificatore, approvatore}
Nella prima pagina di ogni documento o verbale è sempre presente un riassunto dei ruoli svolti dai componenti del gruppo, poi esteso all'interno del registro delle modifiche.
La posizione di redattore e verificatore può essere assunta da più membri del gruppo, ciascuno dei quali può anche ricoprire entrambi i ruoli in base al contributo dato nelle differenti versioni di ogni documento.
L'indicazione del nominativo dei membri del gruppo segue sempre l'ordine del nome e successivamente del cognome.

\subsection{Versionamento}
Tutta la documentazione prodotta dovrà essere inserita in un \textit{repository} git presente su GitHub (\href{https://github.com/Project-SWEnergy/doc-latex/}{github.com/Project-SWEnergy/doc-latex}). 
\par
(INSERIRE STRUTTURA REPOSITORY) 
\par
Ogni documento rilasciato dovrà presentare anche un versionamento interno indicato con tre interi positivi nel formato \texttt{X.Y.Z}, dove:
\begin{itemize}
    \item \textbf{X}: da incrementare in fase di verifica e rilascio di un documento, indica l'ultima versione ufficialmente rilasciata;
    \item \textbf{Y}: da incrementare in caso avvenga una sostanziale modifica, creazione o eliminazione di una sezione;
    \item \textbf{Z}: da incrementare in caso avvenga una sostanziale modifica, creazione o eliminazione di una sottosezione a partire dal secondo livello.
\end{itemize}
Ogni documento dovrà essere creato partendo dalla sezione \texttt{0.1.0}, ogni successivo incremento alla versione dovrà essere accompagnato da una nuova riga nella tabella "Registro delle modifiche" che espliciti i cambiamenti effettuati.
Modifiche quali correzioni grammaticali o leggere variazioni nel testo non vanno riportate nel registro delle modifiche e non producono un avanzamento di versione.
L'avanzamento di versione non deve avvenire prima che il contenuto della modifica sia stato verificato.
L'approvazione di un documento avviene solo quando il documento deve essere rilasciato per una specifica fase di avanzamento.

\subsection{Verifica di uno documento}
