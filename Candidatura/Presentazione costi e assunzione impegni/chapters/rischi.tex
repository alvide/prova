\section{Rischi previsti e relative mitigazioni}

Nessun membro del \textit{team} SWEnergy ha mai lavorato in un'azienda, né ha
mai partecipato ad un progetto di tale portata. Considerando questi fattori, i
rischi previsti potrebbero essere differenti da quelli effettivamente incontrati.

Di seguito sono elencati i rischi individuati insieme alle relative misure di mitigazione:
\begin{enumerate}

\item \textbf{Comprensione delle tecnologie:} il gruppo utilizzerà tecnologie 
\textit{open source} dopo un'attenta ricerca, analisi e confronto tra varie opzioni, 
anche se mai utilizzante precedentemente. 
Potrebbero sorgere difficoltà nella comprensione e nell'implementazione di queste 
nuove tecnologie. \\
\textbf{Mitigazione:} il \textit{team} pianifica un periodo dedicato allo studio e 
sperimentazione delle nuove tecnologie prima dell'avvio effettivo dello sviluppo. 
Inoltre, SWEnergy conta sul supporto offerto dall'azienda per risolvere dubbi e problemi tecnici.

\item \textbf{\textit{Privacy}:} la gestione di documenti aziendali e 
informazioni sensibili richiede un'attenzione rigorosa alla sicurezza e alla 
\textit{privacy}. \\
\textbf{Mitigazione:} SWEnergy si informa sulle normative vigenti riguardo alla 
\textit{privacy} e alla sicurezza dei dati, garantendo l'utilizzo di tecnologie che offrono 
un livello adeguato di sicurezza. 
Eventualmente, le tecnologie che non soddisfano i requisiti di sicurezza vengono scartate.

\item \textbf{Complessità del progetto:} la gestione di \textit{Easy Meal} è
un'impresa complessa che potrebbe richiedere più tempo e risorse del previsto. \\
\textbf{Mitigazione:} il \textit{team} suddivide il progetto in fasi o 
\textit{milestone} chiare e 
pianifica attentamente le tempistiche; monitora costantemente il progresso e
adatta il piano, se necessario. 

\item \textbf{Soddisfazione del proponente:} è fondamentale assicurare la soffisfazione 
dell'azienda proponente riguardo ai risultati del progetto. \\
\textbf{Mitigazione:} SWEnergy mantiene un canale di comunicazione aperto con 
l'azienda proponente e organizza incontri regolari per valutare il progresso del progetto. 
Il \textit{team} si assicura di comprendere appieno le aspettative dell'azienda 
e adatta il lavoro di conseguenza.
\end{enumerate}
