\section{Rischi attesi e mitigazioni}

Nessun membro del \textit{team} SWEnergy ha mai lavorato in un'azienda, né ha
mai partecipato ad un progetto di questa entità. Considerando questi fattori, i
rischi che SWEnergy prevede potrebbero essere distanti da quelli in cui
incorrerà. 
Di seguito sono riportati i rischi individuati e le relative mitigazioni:
\begin{enumerate}

\item \textbf{Comprensione delle tecnologie:} il gruppo utilizzerà tecnologie 
\textit{open source} dopo un'apposita ricerca, analisi e confronto tra varie opzioni, 
anche se mai utilizzante precedentemente. 
Potrebbe esserci un potenziare rischio di incontrare  difficoltà nella comprensione e 
nell'implementazione di queste tecnologie. \\
\textbf{Mitigazione:} il \textit{team} pianifica un periodo dedicato allo studio e 
sperimentazione delle nuove tecnologie prima di iniziare lo sviluppo effettivo. Inoltre, SWEnergy
conta sul supporto offerto dall'azienda per risolvere dubbi e problemi tecnici.

\item \textbf{Privacy:} gestire documenti aziendali e 
informazioni sensibili richiede una rigorosa attenzione alla sicurezza e alla 
\textit{privacy}. \\
\textbf{Mitigazione:} SWEnergy si informa in merito alle normative vigenti
e alla
cura della \textit{privacy} e della sicurezza dei dati garantita dalle tecnologie che
utilizzerà; eventualmente scartando quelle che non offrono un livello di
sicurezza adeguato.

\item \textbf{Complessità del progetto:} la gestione di \textit{Easy Meal} è
un'impresa complessa e potrebbe richiedere più tempo e risorse del previsto. \\
\textbf{Mitigazione:} il \textit{team} suddivide il progetto in fasi o 
\textit{milestone} chiare e 
pianifica attentamente le tempistiche; monitora costantemente il progresso e
adatta il piano se necessario. 

\item \textbf{Soddisfazione del proponente:} assicurarsi che l'azienda proponente sia 
soddisfatta dei risultati. \\
\textbf{Mitigazione:} SWEnergy mantiene un canale di comunicazione aperto con 
l'azienda 
proponente e organizza incontri regolari per valutare il progresso del progetto. 
Il \textit{team} si assicura di comprendere appieno le aspettative dell'azienda 
e adatta il lavoro di conseguenza.
\end{enumerate}
