\section{Calendario di massima del progetto}

Per lo sviluppo del progetto \textit{Easy Meal}, SWEnergy ha 
pianificato un periodo complessivo fino al 7 maggio 2024, con la seguente
distribuzione temporale:
\begin{enumerate}
	\item 8 Settimane per il PRC (Preparazione, Ricerca, e Concettualizzazione): 
	questo primo periodo permetterà al \textit{team} di acquisire conoscenze e 
	le competenze necessarie sulle tecnologie coinvolte, nonchè di elaborare una 
    visione concettuale solida del progetto.

	\item 17 Settimane per l'MVP (\textit{Minimum Viable Product}): 
	successivamente, SWEnergy dedicherà 17 settimane all'implementazione del
	\textit{Minimum Viable Product}. Questa fase si concentrerà sugli elementi 
	essenziali per ottenere un prodotto funzionante e testabile, 
	e possibilmente si occuperà anche di alcuni obiettivi opzionali.
\end{enumerate}

\noindent
Questa pianificazione bilancia il tempo destinato alla preparazione e progettazione 
con quello riservato alla realizzazione effettiva del prodotto. 
Fornisce la flessibilità necessaria per adattarsi alle sfide che potrebbero 
emergere durante il processo di sviluppo.
