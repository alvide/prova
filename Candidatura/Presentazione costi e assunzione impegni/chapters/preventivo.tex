\section{Preventivo dei costi}

\begin{table}[H]
	\renewcommand{\arraystretch}{1.5}
	\centering
	\begin{tabular}{l|r|r|r}
		\textbf{Ruolo} & \textbf{Costo orario (\euro/h)} & \textbf{Ore totali (h)} & 
		\textbf{Costo totale(\euro))} \\
		\hline
		Responsabile				&	 30 &  60 &	 1800			\\
		Amministratore				&	 20 &  54 &   1080  		\\
		Analista					&	 25 &  66 &  1650			\\
		Progettista					&	 25 &  114 &  2850			\\
		Programmatore				&	 15 & 144 &  2160			\\
		Verificatore				&	 15 & 162 &  2430			\\
		\hline
		\textbf{Totale Complessivo} &		& 600 &	11970			\\
	\end{tabular}
	\caption{Preventivo dei costi, divisi per ruolo}

\end{table}

Il team SWEnergy ha elaborato un preventivo finanziario e una divisione dei
ruoli per il progetto \textit{Easy Meal}. È importante 
sottolineare che il costo orario fornito dal docente ha permesso di calcolare 
in modo accurato il costo totale del progetto. \\

\noindent
Ciascun membro del \textit{team} di SWEnergy è ben consapevole che è il primo
progetto di questa portata, e 
rimane aperto alla possibilità di regolare i ruoli in base alle esigenze che 
emergeranno durante lo sviluppo. La flessibilità è una risorsa preziosa in un 
contesto di progetto in evoluzione. \\
La divisione dei ruoli e il costo orario consentono di tenere sotto controllo il
\textit{budget}. Questo è fondamentale per garantire che le risorse finanziarie siano 
allocate in modo efficiente e che il progetto sia economicamente sostenibile.
