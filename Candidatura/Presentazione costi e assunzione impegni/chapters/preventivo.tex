\section{Preventivo dei costi}

\begin{table}[H]
	\renewcommand{\arraystretch}{1.5}
	\centering
	\begin{tabular}{l|r|r|r}
		\textbf{Ruolo} & \textbf{Costo orario (\euro/h)} & \textbf{Ore totali (h)} & 
		\textbf{Costo totale(\euro))} \\
		\hline
		Responsabile				&	 30 &  60 &	 1800			\\
		Amministratore				&	 20 &  54 &   1080  		\\
		Analista					&	 25 &  66 &  1650			\\
		Progettista					&	 25 &  114 &  2850			\\
		Programmatore				&	 15 & 144 &  2160			\\
		Verificatore				&	 15 & 162 &  2430			\\
		\hline
		\textbf{Totale Complessivo} &		& 600 &	11970			\\
	\end{tabular}
	\caption{Preventivo dei costi, divisi per ruolo}

\end{table}

Il team SWEnergy ha elaborato un preventivo finanziario e una suddivisione dei
ruoli per il progetto \textit{Easy Meal}. È cruciale  
sottolineare che il costo orario fornito dal docente ha permesso un calcolo 
preciso del costo totale del progetto. \\

\noindent
Ogni membro del \textit{team} SWEnergy comprende appieno che questo progetto 
rappresenta la prima esperienza di tale portata. Pertanto, si mantiene aperto 
a eventuali aggiustamenti dei ruoli in base alle esigenze emergenti durante lo sviluppo. 
La flessibilità rappresenta un'importante risorsa in un contesto di progetto in costante evoluzione. \\
La distribuzione dei ruoli e il costo orario sono strumenti cruciali per tenere il \textit{budget}  
sotto controllo. Questo aspetto è di fondamentale importanza per garantire un'efficiente 
allocazione delle risorse finanziarie e assicurare che il progetto sia economicamente sostenibile.
