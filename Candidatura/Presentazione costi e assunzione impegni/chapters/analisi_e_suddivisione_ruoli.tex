\section{Analisi e suddivisione dei ruoli}

Di seguito è riportata una riflettuta analisi dei ruoli del progetto che ciascun
membro del \textit{team} SWEnergy dovrà avere a rotazione.
Questa analisi riporta le motivazioni per cui si hanno scelto determinate suddivisioni 
di ore per ciascun ruolo.
\begin{enumerate}

\item \textbf{Analista:} svolge le attività di analisi dei requisiti, quindi deve capire
il problema, essere in grado di comunicare con chi ha un problema, ovvero il proponente 
o l'utente finale. È un ruolo molto delicato, ma non segue il progetto fino al suo termine.
Bisogna dare un valido numero di ore a questo ruolo, ma non eccessivo dato che già la presentazione 
è chiara ed esaustiva.

\item \textbf{Progettista:} svolge le attività di progettazione. 
Una volta analizzato e capito il problema, il progettista lo trasforma
in soluzione e la attua. Suggerisce quali tecnologie usare per creare l'archietettura.
Il progettista seguono lo sviluppo del progetto, ma non la sua manutenzione.
Secondo noi questo ruolo sarà un punto chiave in questo progetto, soprattutto dovuto dal fatto 
che dovremmo cercare tecnologie open source da utilizzarle, confrontandone vantaggi e svantaggi 
per poi scegliere le migliori.

\item \textbf{Programmatore:} svolge le attività di codifica.
Inizia a comparire questa figura non appena c'è una soluzione nota del problema e stanno 
nel progetto finchè non è stato consegnato.
È un esecutore, infatti il progettista propone un \textit{design} il più dettagliato possibile 
per lasciare meno libertà al programmatore che dovrà rispettarlo.
Non è molto semplice, in quanto si deve occupare della codifica dei requisitivi in codice sorgente
eseguibile, conoscere vari linguaggi di programmazione e web, tecniche di ottimizzazione del software 
e le tecnologie decise dal progettista. 


\item \textbf{Verificatore:} svolge le attività di verifica.
È presente per tutto il progetto, andando a verificare ogni singola attività nell'avanzamento del progetto, 
fornendo così una visione oggettiva.
Il verificatore sarà un altro ruolo fondamentale all'interno del progetto a cui saranno assegnate 
un buon numero di ore, dovuto anche dal fatto che nei requisiti proposti nel capitolato è richiesta una 
copertura di test $\geq$ 80\% di report.

\item \textbf{Amministratore:} assicura l'efficienza di procedure, strumenti e tecnologie a supporto del 
\textit{Way of Working}.
È presente per tutto il progetto, ma con frequenza ridotta in quanto funge da supporto per garantire 
stabilità e qualità al progetto. Dovrà avere una conoscenza generica delle tecnologie, e un ottima 
conoscenza delle procedure da utilizzare stabilite nel \textit{Way of Working}.

\item \textbf{Responsabile:} coordina l'elaborazione di piani e scadenze; approva il rilascio di prodotti 
parziali o finali (SW, documenti); coordina le attività di gruppo.
È un ruolo veramente importante, anche se non gli serviranno molte ore nonostante sia presente dall'inizio 
alla fine del progetto.
Deve avere conoscenze varie (tecniche) per essere in grado di comunicare con tutti i ruoli.
È responsabile sulla pianificazione, gestione delle risorse, controllo e coordinamento. 
\end{enumerate}



\textbf{\\Suddivisione dei ruoli:}La suddivisione dei ruoli è stata pensata per renderla completamente equa tra i membri del \textit{team}.
Ciascuno assumera un solo ruolo alla volta per un tempo congruo per poter imparare effettivamente il ruolo stesso 
e far alzare il più possibile le competenze di tutto il gruppo; 
per questo motivo all'inizio la rotazione dei ruoli avverrà ogni 3 settimane per poi passare anche a 2 settimane 
una volta che tutti hanno imparato ciò che effettivamente fa ogni ruolo.