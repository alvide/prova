\section{Analisi e suddivisione dei ruoli}

Di seguito è riportata una riflettuta analisi dei ruoli del progetto che ciascun
membro del \textit{team} SWEnergy dovrà assumere a rotazione.
Tale analisi fornisce le motivazioni alla base delle specifiche allocazioni di ore 
per ciascun ruolo.
\begin{enumerate}

\item \textbf{Analista:} svolge le attività di analisi dei requisiti, pertanto è essenziale comprendere 
il problema e avere la capacità di comunicare con il proponente o l'utente finale. 
Questo ruolo è estremamente delicato, ma non segue il progetto fino al suo completamento.
Assegnare un numero di ore sufficienti a questo ruolo è cruciale, 
tuttavia non eccessivo poiché la presentazione del capitolato è già chiara ed esaustiva.

\item \textbf{Progettista:} svolge le attività di progettazione (\textit{design}). 
Dopo aver analizzato e compreso il problema, il progettista converte tale comprensione in una 
soluzione concreta e la implementa. \\
Inoltre, propone le tecnologie da impiegare per la creazione dell'architettura e supervisiona 
lo sviluppo del progetto, ma non ne gestisce la manutenzione.
Riteniamo che questo ruolo sia di fondamentale importanza in questo progetto, specialmente 
considerando la necessità di individuare e valutare tecnologie \textit{open source} per confrontarne 
i vantaggi e gli svantaggi al fine di selezionare le migliori.

\item \textbf{Programmatore:} responsabile delle attività di codifica. 
Questo ruolo compare non appena una soluzione al problema diventa nota e rimane attivo nel progetto 
fino al suo completamento. 
Svolge il compito di esecutore: seguendo il design dettagliato fornito dal progettista, il programmatore 
ha il compito di trasformare i requisiti in codice sorgente eseguibile. 
Questa mansione non è banale in quanto richiede una profonda conoscenza di diversi linguaggi di programmazione 
e del web, oltre a competenze nelle tecniche di ottimizzazione del \textit{software}. 
Il programmatore deve aderire strettamente alle tecnologie specificate dal progettista, 
che costituiscono la struttura di base su cui costruire il \textit{software}.


\item \textbf{Verificatore:} si occupa delle attività di verifica. 
La sua presenza è costante lungo tutto il progetto, essendo responsabile di esaminare attentamente ciascuna 
fase dell'avanzamento del progetto, fornendo così una valutazione oggettiva. 
Il ruolo del verificatore è di fondamentale importanza all'interno del progetto, richiedendo un'assegnazione 
significativa di ore, specialmente considerando l'esigenza di garantire una copertura dei test pari o superiore 
al 80\% con relativi \textit{report}, come specificato nei requisiti indicati nel capitolato.

\item \textbf{Amministratore:} assicura l'efficienza di procedure, strumenti e tecnologie a supporto del 
\textit{Way of Working}.
Presente per l'intera durata del progetto, sebbene con interazioni meno frequenti, svolge un ruolo di 
supporto cruciale per garantire stabilità e qualità al progetto. 
È richiesta una conoscenza generale delle tecnologie, unitamente a una solida comprensione delle 
procedure stabilite all'interno del \textit{Way of Working}.

\item \textbf{Responsabile:} coordina l'elaborazione di piani e scadenze; approva il rilascio di prodotti 
parziali o finali (\textit{software}, documenti); coordina le attività di gruppo.
Si tratta di un ruolo di grande importanza, anche se potrebbe non richiedere un'elevata quantità di ore 
nonostante la sua presenza costante dall'inizio alla fine del progetto.
Il responsabile deve avere una vasta gamma di conoscenze, sia in ambito tecnico che relazionale, 
al fine di comunicare in modo efficace non solo con gli altri ruoli all'interno del progetto, 
ma anche con soggetti esterni, in quanto rappresenta il progetto verso l'esterno.
Ha la responsabilità della pianificazione, gestione delle risorse, controllo e coordinamento del progetto."
\end{enumerate}



\textbf{\\Suddivisione dei ruoli:}La distribuzione dei ruoli è stata progettata per garantire una completa 
equità tra i membri del \textit{team}. 
Ogni membro assumerà un singolo ruolo per un periodo di tempo adeguato al fine di apprendere in modo efficace 
le responsabilità associate, promuovendo lo sviluppo delle competenze di tutto il gruppo. 
Inizialmente, la rotazione dei ruoli avverrà ogni 3 settimane, successivamente verrà ridotta a 2 settimane 
una volta che tutti i membri hanno acquisito una comprensione completa delle mansioni di ciascun ruolo.