\section{Preventivo dei costi}

\begin{table}[H]
	\renewcommand{\arraystretch}{1.5}
	\centering
	\begin{tabular}{l|l|l|l}
		\textbf{Ruolo} & \textbf{Costo orario} & \textbf{Ore totali} & 
		\textbf{Costo totale} \\ \hline
		\toprule
		Responsabile				& 30€/h & 95h  & 2850€			\\
		Amministratore				& 20€/h & 45h  & 900€				\\
		Analista					& 25€/h & 75h  & 1875€			\\
		Progettista					& 25€/h & 89h  & 2225€			\\
		Programmatore				& 15€/h & 160h & 2400€			\\
		Verificatore				& 15€/h & 100h & 1500€			\\
		\midrule
		\textbf{Totale Complessivo} &	  & 564h & 11750€			\\
	\end{tabular}
	\caption{Preventivo dei costi, divisi per ruolo}

\end{table}

Il team SWEnergy ha elaborato un preventivo finanziario e una divisione dei
ruoli per il progetto \textit{Knowledge management AI}. È importante 
sottolineare che il costo orario fornito dal docente ha permesso di calcolare 
in modo accurato il costo totale del progetto. 
Ciascun membro del \textit{team} di SWEnergy è ben consapevole che è il primo
progetto di questa portata, e 
rimane aperto alla possibilità di regolare i ruoli in base alle esigenze che 
emergeranno durante lo sviluppo. La flessibilità è una risorsa preziosa in un 
contesto di progetto in evoluzione.
La divisione dei ruoli e il costo orario consentono di tenere sotto controllo il
budget. Questo è fondamentale per garantire che le risorse finanziarie siano 
allocate in modo efficiente e che il progetto sia economicamente sostenibile. \\
In sintesi, SWEnergy è consapevole che questa sia una sfida impegnativa, ma
siamo determinati a consegnare un prodotto di alta qualità nel rispetto del
budget e dei tempi previsti. 
