\section{Calendario di massima del progetto}

Per lo sviluppo del progetto \textit{Knowledge management AI} SWEnergy ha 
pianificato un periodo complessivo fino al 29 marzo, con una distribuzione 
temporale come segue:

\begin{enumerate}
	\item 6 Settimane di PRC (Preparazione, Ricerca, e Concettualizzazione): 
	questo primo periodo consentirà al \textit{team} di acquisire conoscenze e 
	competenze necessarie sulle tecnologie coinvolte e di elaborare una solida 
	visione concettuale del progetto.

	\item 14 Settimane di MVP (\textit{Minimum Viable Product}): 
	successivamente, SWEnergy dedicherà 14 settimane all'implementazione del
	\textit{Minimum Viable Product}. Questa fase consentirà al \textit{team} di 
	concentrarsi sull'essenziale per ottenere un prodotto funzionante e 
	testabile; e possibilmente anche su qualche obiettivo opzionale.
\end{enumerate}

Questa pianificazione ci permette di bilanciare il tempo dedicato alla 
preparazione e alla progettazione con quello riservato all'effettiva 
realizzazione del prodotto. Ci darà la flessibilità di adattarci alle sfide che 
potrebbero emergere durante il processo di sviluppo.
