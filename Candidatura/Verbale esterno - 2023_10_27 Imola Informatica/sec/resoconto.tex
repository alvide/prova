\section{Resoconto}
La chiamata è iniziata alle 15:00 e si è conclusa alle 15:30 sulla piattaforma Google Meet con i rappresentati di Imola Informatica: Alessandro Staffola. \\

\subsection{Domande}
Precedentemente il gruppo ha contattato l'azienda per organizzare l'incontro, anticipando le domande più tecniche.
Le domande esposte al referente sono le seguenti:
\begin{enumerate}
	\item "Qual è l'assistenza che l'azienda può metterci a disposizione?".
	\item "Ci sono alcune parti della progettazione o della documentazione che vorreste approfondire?".
	\item "Il progetto deve rispettare i principi di accessibilità? Verranno forniti degli strumenti per la verifica dell'accessibilità?".
	\item "Avremo libertà relativa al \textit{design} dell'applicazione?".
	\item "Consigliate qualche libreria per sviluppare l'applicazione? Per esempio si preferisce sviluppare in Swift e Java, oppure consigliate qualche libreria per sviluppare l'applicazione una volta sola, ad esempio Flutter?".
	\item "Desiderate differenziare l'ordinazione delle pietanze in fase di prenotazione da quella effettuata direttamente al tavolo?".
\end{enumerate}

\subsection{Esito dell'incontro}
Il gruppo è rimasto molto soddisfatto della disponibilità dimostrata nel rispondere alle domande. \\
Gli obiettivi del progetto erano già stati precedentemente espressi con chiarezza, la conversazione è stata incentrata sull'individuazione di possibili criticità e sulla valutazione delle possibili tecnologie da adoperare per lo sviluppo. \\
In particolare è stata posta l'attenzione sulle difficoltà che potrebbero insorgere nella realizzazione di una \textit{chat}, sia dal punto di vista dell'implementazione che della tutela della \textit{privacy} degli utenti.

\noindent
Di seguito sono riassunte le risposte alle domande:
\begin{enumerate}
	\item L'azienda si rende disponibile per incontri settimanali da organizzare secondo necessità. \'E stata presentata anche la possibilità di effettuare incontri di formazione sulle tecnologie scelte dal gruppo.
	\item Oltre a quanto indicato nella presentazione del capitolato non è necessario approfondire alcun aspetto relativo alla progettazione, allo stesso modo non sono richieste aggiunte nella documentazione.
	\item Non viene richiesto un approfondimento sul tema dell'accessibilità.
	\item Il gruppo avrà totale libertà nella scelta del \textit{design} dell'applicazione in tutti i suoi aspetti. Gli elementi presenti nella documentazione del capitolato, come ad esempio le icone, potranno essere utilizzati, modificati o ridisegnati in base alla preferenza del gruppo.
	\item Non sono state indicate preferenze in merito alle tecnologie da utilizzare. Trattandosi della realizzazione di una \textit{web application} è stato sconsigliato al gruppo di utilizzare tecnologie che necissitino di sviluppare più volte l'applicazione come Swift e Java, lasciando però piena libertà sulla decisione finale.
	\item Non è necessario separare le due azioni, l'utente potrà quindi scegliere se effettuare una ordinazione al momento della prenotazione oppure una volta recatosi nel ristorante. 
\end{enumerate}
Durante la conversazione sono emersi diversi dettagli inerenti le funzionalità dell'applicazione come, ad esempio:
\begin{itemize}
	\item la possibilità di accettare una prenotazione in modo automatico in caso il ristorante disponga delle risorse necessarie per la richiesta;
	\item il rifiuto automatico di una prenotazione in caso non venga accettata entro un determinato intervallo di tempo.
\end{itemize} 
Le decisioni inerenti tali implementazioni sono state interamente lasciate al gruppo.

\noindent
L'incontro si è concluso con una conversazione informale volta a conoscere meglio il profilo dell'azienda proponente e del suo referente.


\section{Conclusioni}
Al termine dell'incontro le informazioni raccolte verranno riorganizzate e discusse dai membri del gruppo. \\
Il confronto e le sue conclusioni saranno riportate nel Verbale interno 05, del 27/10/2023.

